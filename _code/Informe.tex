% Options for packages loaded elsewhere
\PassOptionsToPackage{unicode}{hyperref}
\PassOptionsToPackage{hyphens}{url}
%
\documentclass[
]{article}
\usepackage{amsmath,amssymb}
\usepackage{iftex}
\ifPDFTeX
  \usepackage[T1]{fontenc}
  \usepackage[utf8]{inputenc}
  \usepackage{textcomp} % provide euro and other symbols
\else % if luatex or xetex
  \usepackage{unicode-math} % this also loads fontspec
  \defaultfontfeatures{Scale=MatchLowercase}
  \defaultfontfeatures[\rmfamily]{Ligatures=TeX,Scale=1}
\fi
\usepackage{lmodern}
\ifPDFTeX\else
  % xetex/luatex font selection
\fi
% Use upquote if available, for straight quotes in verbatim environments
\IfFileExists{upquote.sty}{\usepackage{upquote}}{}
\IfFileExists{microtype.sty}{% use microtype if available
  \usepackage[]{microtype}
  \UseMicrotypeSet[protrusion]{basicmath} % disable protrusion for tt fonts
}{}
\makeatletter
\@ifundefined{KOMAClassName}{% if non-KOMA class
  \IfFileExists{parskip.sty}{%
    \usepackage{parskip}
  }{% else
    \setlength{\parindent}{0pt}
    \setlength{\parskip}{6pt plus 2pt minus 1pt}}
}{% if KOMA class
  \KOMAoptions{parskip=half}}
\makeatother
\usepackage{xcolor}
\usepackage[margin=1in]{geometry}
\usepackage{graphicx}
\makeatletter
\def\maxwidth{\ifdim\Gin@nat@width>\linewidth\linewidth\else\Gin@nat@width\fi}
\def\maxheight{\ifdim\Gin@nat@height>\textheight\textheight\else\Gin@nat@height\fi}
\makeatother
% Scale images if necessary, so that they will not overflow the page
% margins by default, and it is still possible to overwrite the defaults
% using explicit options in \includegraphics[width, height, ...]{}
\setkeys{Gin}{width=\maxwidth,height=\maxheight,keepaspectratio}
% Set default figure placement to htbp
\makeatletter
\def\fps@figure{htbp}
\makeatother
\setlength{\emergencystretch}{3em} % prevent overfull lines
\providecommand{\tightlist}{%
  \setlength{\itemsep}{0pt}\setlength{\parskip}{0pt}}
\setcounter{secnumdepth}{-\maxdimen} % remove section numbering
\ifLuaTeX
\usepackage[bidi=basic]{babel}
\else
\usepackage[bidi=default]{babel}
\fi
\babelprovide[main,import]{spanish}
% get rid of language-specific shorthands (see #6817):
\let\LanguageShortHands\languageshorthands
\def\languageshorthands#1{}
\ifLuaTeX
  \usepackage{selnolig}  % disable illegal ligatures
\fi
\IfFileExists{bookmark.sty}{\usepackage{bookmark}}{\usepackage{hyperref}}
\IfFileExists{xurl.sty}{\usepackage{xurl}}{} % add URL line breaks if available
\urlstyle{same}
\hypersetup{
  pdftitle={Trabajo de Investigación},
  pdflang={es-ES},
  hidelinks,
  pdfcreator={LaTeX via pandoc}}

\title{Trabajo de Investigación}
\usepackage{etoolbox}
\makeatletter
\providecommand{\subtitle}[1]{% add subtitle to \maketitle
  \apptocmd{\@title}{\par {\large #1 \par}}{}{}
}
\makeatother
\subtitle{Maestría en Ciencia de Datos}
\author{Docente:\\
MSc. Alvaro Limber Chirino Gutierrez}
\date{Maestrantes:\\
- Ada Milenka Cuento Callisaya\\
- Alexander Ernesto Sucre Casas\\
\strut \\
Noviembre, 2023}

\begin{document}
\maketitle

{
\setcounter{tocdepth}{3}
\tableofcontents
}
\pagebreak

\hypertarget{introducciuxf3n}{%
\section{Introducción}\label{introducciuxf3n}}

\hypertarget{objetivos}{%
\section{Objetivos}\label{objetivos}}

\hypertarget{motivaciuxf3n}{%
\section{Motivación}\label{motivaciuxf3n}}

\hypertarget{marco-teuxf3rico}{%
\section{Marco teórico}\label{marco-teuxf3rico}}

\hypertarget{descripciuxf3n-de-la-base-de-datos}{%
\section{Descripción de la base de
datos}\label{descripciuxf3n-de-la-base-de-datos}}

\hypertarget{metodologuxeda}{%
\section{Metodología}\label{metodologuxeda}}

\hypertarget{indicadores-y-fichas-de-indicadores}{%
\subsection{1. Indicadores y fichas de
indicadores}\label{indicadores-y-fichas-de-indicadores}}

\hypertarget{tratamiento-sobre-la-base-de-datos}{%
\subsection{2. Tratamiento sobre la base de
datos}\label{tratamiento-sobre-la-base-de-datos}}

\hypertarget{visualizaciuxf3n-de-la-informaciuxf3n}{%
\subsection{3. Visualización de la
información}\label{visualizaciuxf3n-de-la-informaciuxf3n}}

\hypertarget{armado-de-la-plataforma}{%
\subsection{4. Armado de la plataforma}\label{armado-de-la-plataforma}}

\hypertarget{resultados-y-anuxe1lisis}{%
\section{Resultados y análisis}\label{resultados-y-anuxe1lisis}}

\hypertarget{conclusiones-y-recomendaciones}{%
\section{Conclusiones y
recomendaciones}\label{conclusiones-y-recomendaciones}}

\hypertarget{referencias}{%
\section{Referencias}\label{referencias}}

\end{document}
